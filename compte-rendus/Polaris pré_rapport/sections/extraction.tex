\section{Extraction des étoiles et de leurs coordonnées}
\subsection{Etat de l'art}

Nous sommes dans un premier concentré sur les différents algorithmes et méthodes déjà implémentées pour l'extraction d'étoiles et le traitement d'image.

\vspace{0.5cm}

\noindent \textbf{ICP (Iterative Closest Point) pour la correspondance de nuages de points}

Étape 1 : Correspondance des points. Recherche le point le plus proche dans le nuage cible. Créer des paires de points correspondants entre les deux ensembles.

Étape 2 : Calcul de la transformation. Une transformation géométrique (rotation + translation) est calculée pour minimiser la distance entre les points correspondants. (minimisation de l’erreur quadratique).

Étape 3 : Application de la transformation. 
La transformation calculée est appliquée aux points du nuage source.

Étape 4 : Convergence. Ces étapes sont répétées jusqu'à ce que la différence entre les itérations successives soit inférieure à un seuil (l'algorithme a convergé).

\vspace{0.5cm}

\noindent \textbf{SIFT (Scale-invariant feature transform) pour l'image matching}

Détecter et décrire les features dans chaque image.
Features = zones d’intérêt = intéressantes où se trouvent des formes remarquables.
Comment les détecter? → Détecteur de Moravec
On s’intéresse à un certain pixel de l’image et on regarde les changements d’intensité lumineuse dans son voisinage (Dérivées partielles de I)
On décale ensuite le voisinage dans 2+ directions et on fait la différence des intensités lumineuses avec le voisinage local. Si différence assez grande, alors on a trouvé notre feature = notre étoile.

\vspace{0.5cm}

\noindent \textbf{Top Hat pour l'extraction de points lumineux}

La transformation Top Hat est une opération qui permet d'extraire de petits éléments et détails d'images données. Il existe deux types de transformation top-hat : la transformation white top-hat est définie comme la différence entre l'image d'entrée et son ouverture par un élément structurant, tandis que la transformation black top-hat est définie comme la différence entre l'image de fermeture et l'image d'entrée. Les transformées top-hat sont utilisées pour diverses tâches de traitement d'images, telles que l'extraction de caractéristiques, l'égalisation d'arrière-plan, l'amélioration d'images, etc.

\subsection{Solution choisie}

\subsubsection{Extraction des étoiles et coordonnées : seulement avec la bibliothèque DAOStarfinder
}

Donne directement coordonnées des barycentres (toujours problèmes du seuil) Permet donc d'obetnir avec un seuil manuel, les coordonnées des étoiles

\begin{figure}[ht]
    \centering
    \includegraphics[width = 0.5\linewidth]{img/DAO.png}
    \label{fig3}
    \caption{You can cite the source in the caption \cite{xkcd}.}
\end{figure}


Le Top Hat paraît suffisant pour extraire les étoiles
Problèmes à résoudre actuellement :
Délimiter clairement les contours d’une étoile ce qui fausse l’établissement des barycentres => incertitudes (pour le moment on règle manuellement)

\begin{figure}[ht]
    \centering
    \includegraphics[width = 0.7\linewidth]{img/Topnaze.png}
    \label{fig4}
    \caption{marche pas \cite{xkcd}.}
\end{figure}

On test avec un seuil choisi manuellement

\begin{figure}[ht]
    \centering
    \includegraphics[width = 0.7\linewidth]{img/Topmarche.png}
    \label{fig5}
    \caption{marche \cite{xkcd}.}
\end{figure}

On doit automatiser tout ca