\section{Introduction}
\subsection{Objectifs}
\subsubsection{Préambule}
Développement d’une application dédiée à la reconnaissance automatique des constellations dans le ciel nocturne, en plaçant cette fonctionnalité au cœur de l’expérience utilisateur.
\subsubsection{Objectifs du projet}
\paragraph{Reconnaissance de Constellations} 
Reconnaissance automatique des constellations à partir des coordonnées des étoiles visibles et d’algorithmes d’analyse spatiale (pattern matching)
\paragraph{Interface}
Affichage des constellations identifiées sur une carte simplifiée, enrichie d’informations de base (nom, description, histoire)

\subsection{Contexte}
\subsubsection{Astrométrie et Alignement}
Lors de l'observation astronomique, il est crucial de savoir précisément quelles parties du ciel sont observées. Cela permet de corréler les observations avec des catalogues d'étoiles.

\noindent But : Identifier les étoiles dans une image pour déterminer la position et l'orientation d'un télescope ou d'une caméra.

\noindent Application : Alignement automatique de télescopes amateurs ou professionnels, calibration des instruments, et création de cartes précises du ciel.

\subsection{Plan}  

Nous avons choisi de répartir le travail sur notre projet de la façon suivante :

\begin{figure}[ht]
    \centering
    \includegraphics[width = 0.9\linewidth]{img/resolution.png}
    \label{fig:example_1}
\end{figure}
