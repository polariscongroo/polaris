\section{Introduction}
\subsection{Objectifs}
\textbf{Reconnaissance de Constellations}

\noindent Développement d’une application dédiée à la reconnaissance automatique des constellations dans le ciel nocturne, en plaçant cette fonctionnalité au cœur de l’expérience utilisateur.

\noindent Reconnaissance automatique des constellations à partir des coordonnées des étoiles visibles et d’algorithmes d’analyse spatiale (pattern matching).

\noindent \textbf{Interface}

\noindent Affichage des constellations identifiées sur une carte simplifiée, enrichie d’informations de base (nom, description, histoire).

\subsection{Contexte}
\textbf{Astrométrie et Alignement}

\noindent Lors de l'observation astronomique, il est crucial de savoir précisément quelles parties du ciel sont observées. Cela permet de corréler les observations avec des catalogues d'étoiles.

\noindent But : Identifier les étoiles dans une image pour déterminer la position et l'orientation d'un télescope ou d'une caméra.

\noindent Application : Alignement automatique de télescopes amateurs ou professionnels, calibration des instruments, et création de cartes précises du ciel.

\subsection{Plan}  

Nous avons choisi de répartir le travail sur notre projet de la façon suivante :

\subsubsection{Reconnaissance de Constellations}

\begin{itemize}
    \item \textbf{Organisation}
\end{itemize}

\begin{figure}[ht]
    \centering
    \includegraphics[width = 0.8\linewidth]{img/resolution.png}
    \label{fig1}
    \caption{You can cite the source in the caption \cite{xkcd}.}
\end{figure}

\begin{itemize}
    \item \textbf{Diagramme de classes}
    \begin{itemize}
        \item Créer
    \end{itemize}
\end{itemize}

\subsubsection{Interface}

\begin{itemize}
    \item \textbf{Organisation}
\end{itemize}

\begin{figure}[ht]
    \centering
    \includegraphics[width = 0.5\linewidth]{img/interface.png}
    \label{fig2}
    \caption{You can cite the source in the caption \cite{xkcd}.}
\end{figure}

\begin{itemize}
    \item \textbf{Diagramme de classes}
    \begin{itemize}
        \item Créer
    \end{itemize}
\end{itemize}

\subsection{Recette}

La recette de notre projet est en trois parties :

\begin{itemize}
    \item \textbf{Extraction d'étoiles}
    \begin{itemize}
        \item Traiter la photo prise
        \item Fournir les coordonnées précises des étoiles.
        \item Créer un fichier json pour la partie correspondance
    \end{itemize}

    \item \textbf{Correspondance avec la base de données}
    \begin{itemize}
        \item Créer des traingles avec de plus en plus de points, du plus au moins brillants
        \item Renvoyer le nom de la constellation la plus similaire présente dans la base de données
    \end{itemize}

    \item \textbf{Interface}
    \begin{itemize}
        \item Affiche l'image correpondante à la photographie dans la base de données
        \item Texte original décrivant succintement la constellation
    \end{itemize}
\end{itemize}
