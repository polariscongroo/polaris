\section{Introduction}
\subsection{Objectifs}
\textbf{Reconnaissance de Constellations}

\noindent Developpement d’une application dediee à la reconnaissance automatique des constellations dans le ciel nocturne, en plaçant cette fonctionnalite au cœur de l’experience utilisateur.

\noindent Reconnaissance automatique des constellations à partir des coordonnees des etoiles visibles et d’algorithmes d’analyse spatiale (pattern matching).

\noindent \textbf{Interface}

\noindent Affichage des constellations identifiees sur une carte simplifiee, enrichie d’informations de base (nom, description, histoire).

\subsection{Contexte}
\textbf{Astrometrie et Alignement}

\noindent Lors de l'observation astronomique, il est crucial de savoir precisement quelles parties du ciel sont observees. Cela permet de correler les observations avec des catalogues d'etoiles.

\noindent But : Identifier les etoiles dans une image pour determiner la position et l'orientation d'un telescope ou d'une camera.

\noindent Application : Alignement automatique de telescopes amateurs ou professionnels, calibration des instruments, et creation de cartes precises du ciel.

\subsection{Plan}  

Nous avons choisi de repartir le travail sur notre projet de la façon suivante :

\subsubsection{Reconnaissance de Constellations}

\begin{figure}[ht]
    \centering
    \includegraphics[width = 0.8\linewidth]{img/resolution.png}
    \label{fig1}
    \caption{Susceptible de changer}
\end{figure}

\begin{figure}[ht]
    \centering
    \includegraphics[width = 0.8\linewidth]{img/classreco.jpeg}
    \caption{Diagramme Reconnaissance}
\end{figure}

\subsubsection{Interface}

\begin{figure}[ht]
    \centering
    \includegraphics[width = 0.5\linewidth]{img/interface_img.png}
    \caption{Interface v.0.1}
\end{figure}

\begin{figure}[ht]
    \centering
    \includegraphics[width = 0.6\linewidth]{img/classesInterface.png}
    \caption{Diagramme Interface}
\end{figure}

\subsection{Outils}
\begin{itemize}
    \item\textbf{Repo GitHub} pour notre projet permettant la collaboration et la gestion du code source au sein du groupe.

    \item\textbf{Python} pour la detection d'etoile (traitement d'image). Notamment utilisation de la bibliothèque DAOstarfindfer qui est une bibliothèque adaptee à notre problematique.

    \item\textbf{Java} pour la reconnaissance de constellation et interface : langage rapide et performant.

    \item\textbf{IDE : VScode} pour plus de flexibilte pour passer d'un langague à un autre  

    \item\textbf{IDE : Apache Netbeans} pour l'interface Java Swing.
\end{itemize}

\subsection{Recette}

La recette de notre projet est en trois parties :

\begin{itemize}
    \item \textbf{Extraction d'etoiles}
    \begin{itemize}
        \item Traiter la photo prise : etablir un seuil suffisant qui puisse extraire les etoiles l'ont souhaite etudiees
        \item Fournir les coordonnees precises des etoiles :renvoyer un tableau de coordonnees de chauqe etoiles avec sa luminosite associee 
        \item Renvoyer le resulat dans un format exploitable pour l'etape de correspondance des donnees 
    \end{itemize}

    \item \textbf{Correspondance avec la base de donnees}
    \begin{itemize}
        \item Creer des triangles à partir de points
        \item Renvoyer le nom de la constellation la plus similaire presente dans la base de donnees
    \end{itemize}

    \item \textbf{Interface}
    \begin{itemize}
        \item Affiche l'image correpondante à la photographie dans la base de donnees
        \item Texte original decrivant succintement la constellation
    \end{itemize}
\end{itemize}
\subsection{Gestion de projet}
\textbf{Diagramme de Gantt}
\begin{figure}
    \centering
    \includegraphics[width=0.7\linewidth]{img/Gantt.png}
    \caption{Tâches}
\end{figure}

\begin{figure}
    \centering
    \includegraphics[width=1\linewidth]{img/Gantt2.png}
    \caption{Diagramme de Gantt}
\end{figure}