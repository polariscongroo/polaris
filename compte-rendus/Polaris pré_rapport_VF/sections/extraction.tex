\section{Extraction des etoiles et de leurs coordonnees}
\subsection{Etat de l'art}

Nous sommes dans un premier concentre sur les differents algorithmes et methodes dejà implementees pour l'extraction d'etoiles et le traitement d'image.

\vspace{0.5cm}

\noindent \textbf{ICP (Iterative Closest Point) pour la correspondance de nuages de points}

etape 1 : Correspondance des points. Recherche le point le plus proche dans le nuage cible. Creer des paires de points correspondants entre les deux ensembles.

etape 2 : Calcul de la transformation. Une transformation geometrique (rotation + translation) est calculee pour minimiser la distance entre les points correspondants. (minimisation de l’erreur quadratique).

etape 3 : Application de la transformation. 
La transformation calculee est appliquee aux points du nuage source.

etape 4 : Convergence. Ces etapes sont repetees jusqu'à ce que la difference entre les iterations successives soit inferieure à un seuil (l'algorithme a converge).

\vspace{0.5cm}

\noindent \textbf{SIFT (Scale-invariant feature transform) pour l'image matching}

Detecter et decrire les features dans chaque image.
Features = zones d’interêt = interessantes où se trouvent des formes remarquables.
Comment les detecter? → Detecteur de Moravec
On s’interesse à un certain pixel de l’image et on regarde les changements d’intensite lumineuse dans son voisinage (Derivees partielles de I)
On decale ensuite le voisinage dans 2+ directions et on fait la difference des intensites lumineuses avec le voisinage local. Si difference assez grande, alors on a trouve notre feature = notre etoile.

\vspace{0.5cm}

\noindent \textbf{Top Hat pour l'extraction de points lumineux}

La transformation Top Hat est une operation qui permet d'extraire de petits elements et details d'images donnees. Il existe deux types de transformation top-hat : la transformation white top-hat est definie comme la difference entre l'image d'entree et son ouverture par un element structurant, tandis que la transformation black top-hat est definie comme la difference entre l'image de fermeture et l'image d'entree. Les transformees top-hat sont utilisees pour diverses tâches de traitement d'images, telles que l'extraction de caracteristiques, l'egalisation d'arrière-plan, l'amelioration d'images, etc.

\subsection{Solution choisie}

\subsubsection{Extraction des etoiles et coordonnees : seulement avec la bibliothèque DAOStarfinder
}

Donne directement coordonnees des barycentres (toujours problèmes du seuil) Permet donc d'obetnir avec un seuil manuel, les coordonnees des etoiles

\begin{figure}[ht]
    \centering
    \includegraphics[width = 0.5\linewidth]{img/DAO.png}
    \caption{Detection des points brillant avec seuil, bibliothèque DAOStarfinder}
\end{figure}


Le Top Hat paraît suffisant pour extraire les etoiles
Problèmes à resoudre actuellement :
Delimiter clairement les contours d’une etoile ce qui fausse l’etablissement des barycentres => incertitudes (pour le moment on règle manuellement)

\begin{figure}[ht]
    \centering
    \includegraphics[width = 0.7\linewidth]{img/Topnaze.png}
    \caption{marche pas}
\end{figure}

On test avec un seuil choisi manuellement

\begin{figure}[ht]
    \centering
    \includegraphics[width = 0.7\linewidth]{img/Topmarche.png}
    \caption{marche}
\end{figure}

On doit automatiser tout ca