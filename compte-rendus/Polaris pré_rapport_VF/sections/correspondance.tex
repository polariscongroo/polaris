\section{Correspondance des constellations avec la base de donnee}
\subsection{Etat de l'art}
Pour la correspondance avec la base de donnees, la solution de l'IA etant trop necessiteuse en donnees et pas forcement adaptee à notre cas, nous avons decider de calculer une fonction de coût qui permet de distinguer les caracteristiques des constellations.
\subsection{Solution choisie}

\vspace{0.5cm}

\noindent \textbf{Triangles similaires}

L'algorithme des triangles similaires : (Postulat actuel - La première etape est ideale)
Comparaison des motifs geometriques formes par des groupes d'etoiles. Exploite l'invariance des proportions des triangles.

1. Triangulation : On forme des triangles à partir des etoiles visibles dans l'image capturee (par exemple, à partir des plus brillantes).

2. Caracterisation : Chaque triangle est defini par des ratios de distances qui sont invariants sous rotation, translation et mise à l'echelle.

3. Base de donnees : Ces caracteristiques sont comparees à une base de donnees contenant des triangles pre-calcules à partir de catalogues d'etoiles connues.

4. Correspondence : Si une correspondance est trouvee, fonction de coût.

ABCD : constellation prise en photo; EFGH et IJKL : constellations de la base de donnee

\vspace{1cm}

\noindent \textbf{Exemple}

Determinons la constellation la plus proche de ABCD

\vspace{1cm}
Triangles formes par les points ABCD : 

\noindent [Triangle((0.0, 0.0), (1.0, 0.0), (0.0, 1.0)), Triangle((0.0, 0.0), (1.0, 0.0), (1.0, 1.0)), Triangle((0.0, 0.0), (0.0, 1.0), (1.0, 1.0)), Triangle((1.0, 0.0), (0.0, 1.0), (1.0, 1.0))]
\vspace{1cm}

Triangles formes par les points EFGH : 

\noindent [Triangle((1.0, 1.0), (0.0, 1.0), (1.0, 0.0)), Triangle((1.0, 1.0), (0.0, 1.0), (0.0, 0.0)), Triangle((1.0, 1.0), (1.0, 0.0), (0.0, 0.0)), Triangle((0.0, 1.0), (1.0, 0.0), (0.0, 0.0))]

\vspace{1cm}
Triangles formes par les points IJKL : 

\noindent [Triangle((1.0, 1.0), (0.0, 1.0), (1.0, 0.0)), Triangle((1.0, 1.0), (0.0, 1.0), (10.0, 0.0)), Triangle((1.0, 1.0), (1.0, 0.0), (10.0, 0.0)), Triangle((0.0, 1.0), (1.0, 0.0), (10.0, 0.0))]

\vspace{1cm}
Liste des couts (ABCD - EFGH):

\noindent [0.0, 0.0, 0.0, 0.0, 0.0, 0.0, 0.0, 0.0, 0.0, 0.0, 0.0, 0.0, 0.0, 0.0, 0.0, 0.0]

\noindent Indice des triangles similaires : [0, 1, 2, 3]

\noindent cout total de la constellation : 0

\vspace{1cm}
Liste des couts (ABCD - IJKL) : 

\noindent [0.0, 10.648721674943998, 9.246390843185727, 7.223413489141617, 0.0, 10.648721674943998, 9.246390843185727, 7.223413489141617, 0.0, 10.648721674943998, 9.246390843185727, 7.223413489141617, 0.0, 10.648721674943998, 9.246390843185727, 7.223413489141617]

\vspace{1cm}
\noindent Indice des triangles similaires : [0, 3, 2, 1]

\noindent Cout total de la constellation : 118

\vspace{1cm}
La constellation la plus proche est : 

\noindent [Triangle((1.0, 1.0), (0.0, 1.0), (1.0, 0.0)), Triangle((1.0, 1.0), (0.0, 1.0), (0.0, 0.0)), Triangle((1.0, 1.0), (1.0, 0.0), (0.0, 0.0)), Triangle((0.0, 1.0), (1.0, 0.0), (0.0, 0.0))] donc \textbf{EFGH}

